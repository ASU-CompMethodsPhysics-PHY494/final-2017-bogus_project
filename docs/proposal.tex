\documentclass[11pt, letterpaper]{article}
\usepackage[utf8]{inputenc}
\usepackage{geometry}[1in]
\usepackage{enumitem}
\usepackage{amsmath}

\begin{document}
 
\centerline{\textbf{\Large PHY494 - Spring 2017 - Final Project Proposal}}
\centerline{\large Brian Pickens}
\hfill
\subsection*{Problem}
\paragraph{} The mechanics of quantum physics are strange. One of the (many) difficulties of understanding the quantum world is the reliance of simplified one-dimensional situations to introduce students to a particle's odd wave-like behavior. But we live in a three-dimensional world and I wonder what a 3D particle looks like. How does it interact over time with rigid or dynamic potential barriers? What does quantum tunneling look like in 3D? Can these be modeled accurately by a computer?

\subsection*{Approach}
\paragraph{} The main obstacle to answering these questions will be solving Schr\"odinger's equation in three dimensions. Specifically, we will be attempting to use the real/imaginary position integration scheme by Maestri \textit{et al}.:
\begin{multline}
R^{n+1}_{i,j,k} = R^{n-1}_{i,j,k} + 2[(4\alpha + \frac{1}{2}\Delta tV_{i,j,k})I^n_{i,j,k} - \alpha (I^{n}_{i+1,j,k} + I^{n}_{i-1,j,k}\\ + I^{n}_{i,j+1,k} + I^{n}_{i,j-1,k} + I^{n}_{i,j,k+1} + I^{n}_{i,j,k-1})]
\end{multline}
\begin{multline}
I^{n+1}_{i,j,k} = I^{n-1}_{i,j,k} + 2[(4\alpha + \frac{1}{2}\Delta tV_{i,j,k})R^n_{i,j,k} - \alpha (R^{n}_{i+1,j,k} + R^{n}_{i-1,j,k}\\ + R^{n}_{i,j+1,k} + R^{n}_{i,j-1,k} + R^{n}_{i,j,k+1} + R^{n}_{i,j,k-1})]
\end{multline}
\paragraph{} To visualize the particle's behavior, we will model the \textit{probability} of detecting the particle at any position. The value of the probability at a data point will be calculated from the above integration scheme. Visualizing a probability density in 3D can be difficult, so we'll be using the yt package which can be installed into the Anaconda distribution of Python.

\subsection*{Objectives}

\begin{enumerate}[noitemsep, topsep=0pt, parsep=0pt, partopsep=0pt]
\item Simulate the time-dependent behavior of a Gaussian wave packet in 2D using initial conditions of $\Psi(0, t) = \Psi(L, t) = 0$ and $V(x, t) = \infty$ for $x < 0$ and $x > L$.
\item Test algorithm by comparing 2D results to the analytical predictions of phenomena such as quantum revival and quantum number degeneracies.
\item Further test the algorithm by tracking the total integrated probabilty over time.
\item Move to 3D particle simulation enclosed both in a box and in an isotropic 3D harmonic oscillator, simulated as three independent oscillators in each axis.
\item BONUS: Implement 3D quantum tunneling effects for potential barriers.
\end{enumerate}

\end{document}